% !TeX spellcheck = en_GB

\documentclass[proc]{edpsmath}
\usepackage[utf8]{inputenc}
\usepackage[T1]{fontenc}

\usepackage{amsmath,amssymb,amsthm}
\usepackage{graphicx}
\usepackage{hyperref}
\usepackage{physics}
\usepackage{mathtools}

\newcommand{\bR}{\mathbb{R}}
\newcommand{\bb}{\mathbf{b}}
\newcommand{\bA}{\mathbf{A}}
\newcommand{\vp}{v_\parallel}
\newcommand{\bx}{\mathbf{x}}
\newcommand{\bv}{\mathbf{v}}
\newcommand{\bu}{\mathbf{u}}
\newcommand{\bB}{\mathbf{B}}
\newcommand{\Dt}{\Delta t \;}

\newcommand{\dbx}{\dot{\mathbf{x}}}
\newcommand{\diff}[2]{\frac{\text{d} #1}{\text{d} #2}}
\newcommand{\pa}[2]{\frac{\partial #1}{\partial #2}}
\newcommand{\ap}{a_\parallel}
\newcommand{\mean}[1]{\langle #1 \rangle}
\newcommand{\Bap}{B_\parallel^\ast}
\newcommand{\bracket}[2]{\left\{ #1, #2 \right\}}
\newcommand{\gcbracket}[2]{\left\{ #1, #2 \right\}_{\text{g.c.}}}
\newcommand{\dt}{\frac{\text{d}}{\text{d} t}}

\renewcommand{\d}{\;\text{d}}

%TODO: agreed?
\setlength{\parindent}{0pt}

\begin{document}

\title{SILAS - Semi-Lagrangian scheme with Arakawa splitting}
\thanks{CEMRACS}\thanks{Max-Planck}% At most 5 thanks

\author{Dominik Bell}\address{Max-Planck-Institut für Plasmaphysik, Garching, Germany; \email{dominik.bell@ipp.mpg.de\ \&\ frederik.schnack@ipp.mpg.de\ \&\ martin.campos-pinto@ipp.mpg.de\ \&\ sonnen@ipp.mpg.de}}\secondaddress{Technische Universität München, Zentrum Mathematik, Garching, Germany}
\author{Martin Campos Pinto}\sameaddress{1}
\author{Davor Kumozec}\address{Faculty of Sciences, University of Novi Sad, Serbia; \email{davor.kumozec@dmi.uns.ac.rs}}
\author{Frederik Schnack}\sameaddress{1}
\author{Eric Sonnendrücker}\sameaddress{1,2}


\begin{abstract}
In practice, there are various ways to solve gyro-kinetic equations. Here, we will combine the Semi-Lagrangian (SL) method and the Arakawa (AKW) scheme combined with a time-integrator. Both methods are successfully used in practice when applied individually, in our case, we combine them by first decomposing the problem into a fast (parallel) and a slow (perpendicular) dynamical system. The SL approach will be used to solve the fast and the AKW scheme for the slow subsystem. So far in our reference code \cite{pygyro_code}, the entire model is solved only using the SL method. Our goal is to replace the discretization of the slow subsystem by the AKW scheme, in order to improve preservation of conserved quantities of the physical system at hand. \\
After an introduction of the gyro-kinetic equations in Section \ref{sec:introduction}, we will describe this splitting ansatz in combination with the two schemes in Section \ref{sec:splitting_discretization}. Lastly in Section \ref{sec:num_exp}, numerical experiments will be presented, first as a verification of the AKW scheme, and after that the application to the full PyGyro model. Concluding with Section \ref{sec:conclusion}, we discuss the benefits of this approach. 
\end{abstract}

\begin{resume}%TODO: verify french
	En pratique, il existe plusieurs façons de résoudre les équations gyro-cinétiques. Ici, nous combinerons la méthode semi-lagrangienne (SL) et le schéma d'Arakawa (AKW) combiné à un intégrateur temporel. Les deux méthodes sont utilisées avec succès dans la pratique lorsqu'elles sont appliquées individuellement. Dans notre cas, nous les combinons en décomposant d'abord le problème en un système dynamique rapide (parallèle) et un système dynamique lent (perpendiculaire). L'approche SL sera utilisée pour résoudre le système rapide et le schéma AKW pour le sous-système lent. Jusqu'à présent, dans notre code de référence \cite{pygyro_code}, le modèle entier est résolu en utilisant uniquement la méthode SL. Notre objectif est de remplacer la discrétisation du sous-système lent par le schéma AKW, afin d'améliorer la préservation des quantités conservées du système physique en question. \\
	Après une introduction des équations gyro-cinétiques dans la Section \ref{sec:introduction}, nous décrirons cet ansatz de fractionnement en combinaison avec les deux schémas dans la Section \ref{sec:splitting_discretization}. Enfin, dans la section \ref{sec:num_exp}, des expériences numériques seront présentées, d'abord pour vérifier le schéma AKW, puis pour l'appliquer au modèle PyGyro complet. Dans la section \ref{sec:conclusion}, nous discutons des avantages de cette approche. 
\end{resume}

\maketitle
\newpage
\tableofcontents %TODO: can also be removed later

% !TeX spellcheck = en_GB


\section{The Gyro-kinetic Model}
\label{sec:introduction}

In a Tokamak, particles have complicated dynamics in real space which consists of a slow motion along the magnetic field lines superimposed with a fast motion circling around the magnetic field lines. This fast motion is called gyration and can be averaged out to reduce the dimension of phase space in order to make the models computationally more feasible, while keeping most of the important physics. The resulting theory is called a gyro-kinetic model which will be discussed in the following.\\

The model is defined by the Lie-transformed, low-frequency particle Lagrangian $L$, where we follow the derivation in \cite{Bottino_Sonnendrucker_2015}, \cite{emily} and \cite{Latu_2017}. Given a static magnetic field $\bB$, its intensity $B = \norm{\bB}$ and its direction $\bb = \bB / B$, the particle charge $q \not= 0$ and the particle mass $m >0$,  
%and denoting $\bA$ as the vector-valued background potential
we are able to write the Lagrangian $L$ as
\begin{equation}\label{Lagrangian}
	L(t, \bx, \vp, \mu) = \left(\nabla \times q\bB + m \vp \bb \right)\cdot\dbx + \frac{m}{q} \mu \dot{\theta} - H(\bx, \vp),
\end{equation}
where $\bx \in \Omega \subseteq \bR^3$ is the position of the gyro-centre, $\vp \in \bR$ is the velocity parallel to the magnetic field lines and $\mu$ is the modified magnetic moment. The Hamiltonian $H$ will be introduced shortly.
Looking at the equation of motion of $\theta$, i.e. its Euler-Lagrange equations, we immediately can conclude that $\mu$ is an exact invariant of the system, i.e.
\begin{equation}
	\diff{}{t} \mu = 0,
\end{equation}
and thus the phase-space is only $4$-dimensional.
The gyro-kinetic equation describing the gyro-centre distribution function $f=f(t,\bx, \vp, \mu)$ is given by 
\begin{equation}\label{drift-kinetic model}
	\pa{f}{t} + \bu \cdot \nabla f + \ap \pa{f}{\vp} = 0,
\end{equation}
which describes the positions of a collection of identical particles and whose exact solution is constant along the trajectories $(\bx(t), \vp(t), \mu(t))$ in the phase-space, i.e. 
\begin{equation}
	\frac{\dd}{\dd t} f(t,\bx(t), \vp(t), \mu(t)) = 0.
\end{equation}
In order to determine the equations of motion for ${\dd \bx}/{\dd t} = \bu$ and ${\dd \vp}/{\dd t} = \ap$, we look at the remaining Euler-Lagrange equations and introduce the known electrostatic gyro-center Hamiltonian $H(\bx, \vp)$, which reads
\begin{equation}
	H(t, \bx, \vp, \mu) = \frac{1}{2} m \vp^2 + \mu B(\bx) + q \mean{\phi}_\alpha (t,\bx),
\end{equation}
where the bracket denotes an averaging over the gyro-angle $\alpha$, i.e.
\begin{equation}
	 \mean{\;\cdot\;}_\alpha \coloneqq \frac{1}{2\pi} \int \cdot \ \dd \alpha,
\end{equation}
of the electrostatic potential $\phi = \phi(t, \bx)$. 
Simplifying notations by defining
	\begin{equation}
		\bB^\ast  \coloneqq \bB + \frac{m}{q}\vp \nabla \times \bb , \qquad
		\Bap  \coloneqq \bb \cdot \bB^\ast = B + \frac{m \vp}{q B} \bb \cdot \left( \nabla \times \bB \right),
	\end{equation}
one can derive the characteristic trajectories from the remaining Euler-Lagrange equations of \eqref{Lagrangian}, which yield
\begin{subequations}
	\begin{align}
		\bu  & = \frac{1}{\Bap} \left( \frac{1}{m} \pa{H}{\vp} \bB^\ast + \frac{1}{q} \bb \times \nabla H \right) \label{eom for bu}, \\
		\ap & = \frac{1}{\Bap} \left( -\frac{1}{m} \bB^\ast \cdot  \nabla H  \right). \label{eom for ap}
	\end{align}
\end{subequations}
As noted in \cite{Latu_2017}, the phase space is divergence-free, i.e.
\begin{equation}
	\nabla \cdot \bu + \pa{\ap}{\vp} = 0,
\end{equation}
thus we can rewrite \eqref{drift-kinetic model} in conservative form
\begin{equation}\label{conservation1}
	\pa{}{t}\left(\Bap f\right) + \nabla\cdot\left( \Bap \bu f \right) + \pa{}{\vp} \left( \Bap \ap f \right) = 0.
\end{equation}
Splitting the equations into fast and slow subsystems, i.e. $\bu = \bu_f + \bu_s$ and $a_{\parallel} = a_{\parallel, f} + a_{\parallel, s}$, yields:
\begin{subequations}
	\begin{align}
		(\text{fast}) = & \left\{ \begin{aligned}
			\bu_f & = \frac{1}{\Bap} \frac{1}{m} \pa{H}{\vp} \bB, \\
			a_{\parallel, f} & = -\frac{1}{\Bap} \frac{1}{m} \bB \cdot \left( \nabla H \right),
		\end{aligned} \right. \label{split step fast}\\
		(\text{slow}) = &  \left\{ \begin{aligned}
			\bu_s & = \frac{1}{\Bap} \frac{1}{q} \left( \pa{H}{\vp} \vp \nabla \times \bb + \bb \times \left(\nabla H\right) \right), \\
			a_{\parallel, s} & = - \frac{a}{\Bap} \frac{1}{q} \vp \left( \nabla \times \bb \right) \cdot \left( \nabla H \right).
		\end{aligned} \right. \label{split step slow}
	\end{align}
\end{subequations}
Both fast and slow steps are divergence-free and derived from a Poisson bracket.
This guiding-center Poisson bracket is given by
\begin{equation}
	\begin{aligned}
		\gcbracket{F}{G} & = \frac{\bB^\ast}{m \Bap} \cdot \left( \left(\nabla F\right) \pa{G}{\vp} - \left(\nabla G\right) \pa{F}{\vp} \right) \\
		& \qquad + \frac{\vp}{q \Bap} \left( \nabla \times \bb \right) \cdot \left( \left(\nabla F\right) \pa{G}{\vp} - \left(\nabla G\right) \pa{G}{\vp} \right) \label{gc_lg} \\
		& \qquad - \frac{1}{q \Bap} \bb \cdot \left[\left(\nabla F\right) \times \left(\nabla G\right)\right]. 
	\end{aligned}
\end{equation}
For a constant background magnetic field $\bB$, the model simplifies and the subsystems become
\begin{subequations}
	\begin{align}
		(\text{fast}) = & \left\{ \begin{aligned}
			\bu_f & = {\vp} \bb, \\
			a_{\parallel, f} & = -\frac{q}{m} \bb \cdot \nabla \phi,
		\end{aligned} \right. \label{split step fast const B}\\
		(\text{slow}) = &  \left\{ \begin{aligned}
			\bu_s & = \frac{\nabla \phi\times \bb}{\bB},  \\
			a_{\parallel, s} & = 0.
		\end{aligned} \right. \label{split step slow const B}
	\end{align}
\end{subequations}
Finally, we are able to rewrite the space variables to the screw-pinch model in cylindrical coordinates that is introduced in \cite{Latu_2017}. We look for the distribution function $f = f(t, r, \theta, z, v_\parallel)$ satisfying
%TODO add sources? 
\begin{equation}
\partial_t f + \{\phi, f\} + v_\parallel \nabla_\parallel - \nabla_\parallel \phi \partial_{v_\parallel}f = 0, \label{eq:GK_model}
\end{equation}
where $\nabla_\parallel = \bb \cdot \nabla$ and the bracket is transformed to polar coordinates, which reads
\begin{equation}
\{\phi, f \} = \frac{1}{rB_0}\partial_r \phi\partial_\theta f -\frac{1}{rB_0}\partial_\theta \phi\partial_r f. \label{eq:bracket_in_polar}
\end{equation}
Since the plasma is quasi-neutral, what means that locally there may be charged regions but it is neutral overall, this equation is complemented by solving the Poisson type quasi-neutrality (QN) equation for the self-consistent potential $\phi = \phi(t, r, \phi, z)$, i.e. solving for a given temperature profile $T_e$
\begin{equation}
- \left[\partial_r^2 \phi + \left( \frac{1}{r} + \frac{\partial_r n_0}{n_0}\right)\partial_r \phi + \frac{1}{r^2} \partial_\theta^2 \phi \right] + \frac{1}{T_e} \phi = \int_{-\infty}^{\infty} (f - f_\text{eq}) \ \dd v_\parallel, \label{eq:qn}
\end{equation}
where the given (radial symmetric) equilibrium function $f_\text{eq}$ is a Gaussian and the initial density function $n_0$ is the integral over $\vp$ of the initial distribution function $f(t=0, r, \theta, z, v_\parallel)$ that is the equilibrium function plus some excited modes.


%TODO: Should this be here? Rather in numerical examples
% This equation will be treated using finite element method (FEM) on a B-spline basis. The independence of the equation from the $z$ direction and the periodicity of the solution in $\theta$ direction will be used to parallelize the code.

%\subsection{Boundary Conditions and Conservation Laws}

To finalize this system of equations, we briefly want to discuss the boundary conditions of the screw-pinch model. The distribution function $f$ is periodic along $\theta$, $z$ and $\vp$. In the radial direction $r$, we assume the values are given an outside equilibrium function. For the potential $\phi$, we assume periodic boundary conditions along $\theta$ and $z$. In $r$-direction, we decompose $\phi$ into Fourier modes at $r_\text{min}$, taking homogeneous Neumann boundary conditions for the zeroth mode and homogeneous Dirichlet boundary conditions for the others. At $r_\text{max}$, we just take plain homogeneous Dirichlet boundary conditions.\\


%Does this apply? Can you explain it to me? 
%\subsubsection{Conserved Quantities}
Lastly, we want to take a look at physical properties of these equations. Since we have a transport equation in conservative form, i.e. \eqref{conservation1},that conserves arbitrary functions of $f$  along non-linear characteristic trajectories, we can write the Casimir equation
\begin{equation}
\dt C(f) + \bracket{C(f)}{H} =0,
\end{equation}
where the Casimir invariant $\int C( f ) d^5R$ is conserved. Therefore, in addition to the particle number or the $L^1$-norm $\int f d^5R$, the system has an infinite number of conserved quantities such as the $L^2$-norm $\int f^2 d^5R$, and the kinetic entropy $S=\int f \ln(f) d^5R$.\\
Thirdly, the gyro-kinetic equations conserve the total energy $H = E_k + E_f$:
\begin{subequations}
	\begin{align}
		E_k &= \frac{1}{2} m \vp^2 + \mu B(\bx),\\
		E_f &= q \mean{\phi}_\alpha (t,\bx).
	\end{align}
\end{subequations}
where $E_k$ is the kinetic energy and $E_f$ is the field energy. For more details on this, we refer the reader to \cite{idomura2008conservative}.

% !TeX spellcheck = en_GB

\section{Operator Splitting and Discretization}
\label{sec:splitting_discretization}

\textbf{Strang splitting} is a numerical tool which can be used in numerical computation of PDEs, coupled together with the \textbf{Lie splitting} %TODO Strang splitting and Lie splitting are not seperate tools, are they?
gives us the resulting equations which are much simpler for solving with reasonable error (for more details see [emily thesis] %TODO: Add citation
). We can split our equation 
\begin{equation}
 \partial_t f + \{\phi, f \} + v_\parallel \nabla_\parallel f - \nabla_\parallel \phi\,\, \partial_{v_\parallel} f = 0
\end{equation}
for the distribution function $f$ into three advection equations:
\begin{subequations}
	\begin{align}
		\partial_t f + v_\parallel \nabla_\parallel f & = 0 && \text{Advection on flux surface} & \\
		\partial_t f + \nabla_\parallel \phi\,\, \partial_{v_{\parallel}} f & = 0 && \text{V-parallel advection} & \\
		\partial_t f + \{\phi, f\} & = 0 && \text{Advection on poloidal plane} &
	\end{align}
\end{subequations}
where $\{\phi,f\}$ is a Poisson bracket, defined as follows:
\begin{equation}
 \{\phi,f\}=-\frac{\partial_\theta\phi}{rB_0}\partial_r f + \frac{\partial_r\phi}{rB_0}\partial_\theta f.
\end{equation}
In our work, we will use two different approaches to solving these equations. For the first two we will use the semi-Lagrangian scheme and for the third equation we will use the Arakawa scheme.







\subsection{Semi-Lagrangian Scheme for the Fast Time Subsystem}

The fast time subsystem will be solved using semi-Lagrangian scheme. As reference we will use \cite{campospinto} and [emily thesis]. %TODO: add citation
The code for this part can be found at \cite{pygyro_code}. We will use general notations in the beginning and then move on to the specific problem we have in our model. This scheme is frequently used in solving advection problems of the form:
\begin{equation}
    \partial_t f(t,\mathbf{x})+v(t,\mathbf{x})\cdot \nabla f(t,\mathbf{x})=0, \qquad t\in[0,T], \quad \mathbf{x}\in\mathbb{R}^d
\end{equation}
where $v$ is a velocity field $\mathbb{R}^d\longrightarrow\mathbb{R}^d$, $T$ is a final time, and initial conditions are given by $f_0(\mathbf{x})=f(0,\mathbf{x})$. For simplicity, we will assume that $v$ is given and smooth enough so we can use the method of characteristics. Thus we can obtain the trajectories $X(t)=X(t;s,x)$ as a solutions to ODE
\begin{equation}
    X'(t)=v(t,X(t)), \qquad X(s)=x, \qquad t\in[0,T].
\end{equation}
It can be shown that the flow $F_{s,t}:x\longrightarrow X(t)$ is invertible and satisfies $(F_{s,t})^{-1}=F_{t,s}$. We know that the analytical solution to our equation is given by
\begin{equation}
    f(t,\mathbf{x})=f_0((F_{0,t})^{-1}(\bx)),\qquad \text{for } t\in[0,T], \quad \mathbf{x}\in\mathbb{R}^d.
\end{equation}
Now we are using backward tracking of the characteristic
\begin{equation}
    B^{n,n+1}=(F_{t_n,t_{n+1}})^{-1}
\end{equation}
between two time steps $t_n=n\Delta t$ and $t_{n+1}$. For the last step we have to approximate the function $f$ at the grid points $f^n=f(t_n)$. For this purpose, we will use splines.

The flux surface advection operator defined by the equation
\begin{equation}
 \partial_t f + v_\parallel \nabla_\parallel f = 0
\end{equation}
and is a two-dimensional semi-Lagrangian operator. Here we can determine the exact trajectory because the velocity in the equation is constant and is not related to the surface of the flux.

In order to determine the value of the distribution function in the discretization grid for characteristics that go outside our domain, we use a combination of a one-dimensional cubic spline and a Lagrangian interpolation polynomial of fifth order in $\theta$ and $z$ directions, respectively. In this way, we get an approximation of the value of the function in the final position.

The v$_\parallel$ surface advection operator defined by equation
\begin{equation}
    \partial_t f + \nabla_\parallel \phi\,\, \partial_{v_{\parallel}} f = 0
\end{equation}
and is a one-dimensional semi-Lagrangian operator. Here we use a cubic spline to determine the value of the particle distribution function for characteristics that go outside the domain. The parallel gradient of $\phi$ depends only on the spatial coordinates and is therefore constant along the surface $v_\parallel$. As a result, the trajectory used by the semi-Lagrangian method can be accurately defined. The parallel gradient of $\phi$ is computed using a (field-aligned) finite difference method of order 6 in the $z$ direction. This was calculated as described by Latu et al. \cite{Latu_2017}.





\subsection{Arakawa Scheme for the Slow Time Subsystem}

For the Arakawa scheme, we mainly reference the article \cite{Arakawa_1966}. Here we are interested in solving the advection on poloidal plane:
\begin{equation}
 \partial_t f + \{\phi, f\} = 0
\end{equation}
where $\{\phi,f\}$ is a Poisson bracket, defined as follows:
\begin{equation}
 \{\phi,f\}=-\frac{\partial_\theta\phi}{rB_0}\partial_r f + \frac{\partial_r\phi}{rB_0}\partial_\theta f.
\end{equation}
The Arakawa scheme provably preserves the following quantities: %TODO: add proof
	\begin{align*}
		\text{mass} : && \dt \int f(t) \d x \text{d} y & = 0 & \Leftrightarrow && \int\bracket{\phi}{f} \d x \text{d} y & = 0 \\
		L^2\text{-norm :} && \dt \int f^2(t) \d x \text{d} y & = 0 & \Leftrightarrow && \int f \, \bracket{\phi}{f} \d x \text{d} y & = 0 \\
		\text{total energy :} && \dt \int \phi \, f(t) \d x \text{d} y & = 0 & \Leftrightarrow && \int \phi \bracket{\phi}{f} \d x \text{d} y & = 0
	\end{align*}
First we will focus on discretization of Poisson bracket in Cartesian coordinates. The scheme in polar coordinates is analogous.





\subsubsection{Construction of a Discrete Bracket using the Arakawa Method}

The first step in constructing the 4th order Arakawa scheme is to explain second order discretization using nine point stencil. We have approximation of $J(f,g)$ at $(i,j)$ of the following form
\begin{equation}
J(f,g)=\frac{1}{3}(J^{++}+J^{+\times}+J^{\times+})+\mathcal{O}(d^2)
\end{equation}
where
\begin{equation}
    J^{++}_{ij}=\frac{1}{4d^2}[(f_{i+1,j}-f_{i-1,j})(g_{i,j+1}-g_{i,j-1})-(f_{i,j+1}-f_{i,j-1})(g_{i+1,j}-g_{i-1,j})],
\end{equation}
\begin{equation}
    J^{+\times}_{ij}=\frac{1}{4d^2}[f_{i+1,j}(g_{i+1,j+1}-g_{i+1,j-1})-f_{i-1,j}(g_{i-1,j+1}-g_{i-1,j-1})-f_{i,j+1}(g_{i+1,j+1}-g_{i-1,j+1})+f_{i,j-1}(g_{i+1,j-1}-g_{i-1,j-1})]
\end{equation}
and
\begin{equation}
    J^{\times+}_{ij}=\frac{1}{4d^2}[f_{i+1,j+1}(g_{i,j+1}-g_{i+1,j})-f_{i-1,j-1}(g_{i-1,j}-g_{i,j-1})-f_{i-1,j+1}(g_{i,j+1}-g_{i-1,j})+f_{i+1,j-1}(g_{i+1,j}-g_{i,j-1})]
\end{equation}
We denote the linear combination $J_1=\frac{1}{3}(J^{++}+J^{+\times}+J^{\times+})$, which is a second order approximation conserving the square of vorticity and the energy.

In order to extend the discretization to 4th order we introduce $J_2=\frac{1}{3}(J^{\times \times} + J^{\times+} + J^{+\times})$ where we use the same nine point stencil with the additional
four points $(i+2,j)$, $(i-2,j)$, $(i,j+2)$  and $(i,j-2)$ for $J_2$, where
\begin{equation}
    J^{\times\times}_{ij}=\frac{1}{8d^2}[(f_{i+1,j+1}-f_{i-1,j-1})(g_{i-1,j+1}-g_{i+1,j-1})-(f_{i-1,j+1}-f_{i+1,j-1})(g_{i+1,j+1}-g_{i-1,j-1})], 
\end{equation}
\begin{equation}
    J^{\times+}_{ij}=\frac{1}{8d^2}[f_{i+1,j+1}(g_{i,j+2}-g_{i+2,j})-f_{i-1,j-1}(g_{i-2,j}-g_{i,j-2})-f_{i-1,j+1}(g_{i,j+2}-g_{i-2,j})+f_{i+1,j-1}(g_{i+2,j}-g_{i,j-2})],
\end{equation}
and
\begin{equation}
    J^{+\times}_{ij}=\frac{1}{8d^2}[f_{i+2,j}(g_{i+1,j+1}-g_{i+1,j-1})-f_{i-2,j}(g_{i-1,j+1}-g_{i-1,j-1})-f_{i,j+2}(g_{i+1,j+1}-g_{i-1,j+1})+f_{i,j-2}(g_{i+1,j-1}-g_{i-1,j-1})].
\end{equation}
By Taylor extension, we can see that $2J_1-J_2$ is a fourth order approximation of the Jacobian $J$; that is,
$2J_1-J_2=J+O(d^4)$.
Now we have the second-order nine-point
scheme and the fourth-order thirteen-point scheme. For the proof of the stencil order, we refer to appendix \ref{sec:ara_order}.




\subsubsection{Boundary Conditions}

Let show first what is happening if $g$ is constant at boundary. By
\begin{equation}
    J_{ij}(f,g)=\sum^*_{i',j'}[a_{i,j;i+i',j+j'}(f_{i+i',j+j'}+f_{i,j})-a_{i-i',j-j';i,j}(f_{i,j}+f_{i-i',j-j'})].
\end{equation}
and assume g is constant at the boundary j = 0 (and outside of domain), the discrete Jacobi is treated as
\begin{equation}
\begin{aligned}
\frac{1}{2}J_{i0} (f,g) &= a_{i,0;i+1,0}(f_{i+1,0}+f_{i,0})-a_{i-1,0;i,0}(f_{i-1,0}+f_{i-1,0})\\
&-\frac{1}{12d^2}[(g_{i+1,0}+g_{i+1,1}-g_{i-1,0}-g_{i-1,1})(f_{i,0}+f_{i,1})\\
&+(g_{i+1,0}-g_{i,1})(f_{i,0}+f_{i+1,1})\\
&+(g_{i,1}-g_{i-1,0})(f_{i-1,1}+f_{i,0})].
\end{aligned}
\end{equation}
We need to keep f constant case (stationary state), so it holds that $$\sum_{i',j'} a_{i,j;i+i',j+j'}=0.$$ 
Then 
$$a_{i,0;i+1,0}-a_{i-1,0;i,0}-\frac{1}{12d^2}(2g_{i+1,0}-2g_{i-1,0}+g_{i+1,1}-g_{i-1,1})=0.$$
Rewrite it as 
$$a_{i,0;i+1,0}-\frac{1}{12d^2}(g_{i,1}+g_{i+1,1}-g_{i,0}-g_{i+1,0})=a_{i-1,0;i,0}-\frac{1}{12d^2}(g_{i-1,1}+g_{i,1}-g_{i-1,0}-g_{i,0}).$$
Right hand side has the same formula with left hand side by replacing i by i-1. Each term is not depend on i, so it is constant. Because this term $a_{i,0;i+1,0}(f_{i+1,0}+f_{i,0})-a_{i-1,0;i,0}(f_{i-1,0}+f_{i-1,0})$ is a approximation of $-\partial_y g \partial_x f$ on the boundary, so the constant is 0.
At last, we have $$a_{i,0;i+1,0}=\frac{1}{12d^2}(g_{i,1}+g_{i+1,1}-g_{i,0}-g_{i+1,0})$$ and $$a_{i-1,0;i,0}=\frac{1}{12d^2}(g_{i-1,1}+g_{i,1}-g_{i-1,0}-g_{i,0}).$$ 
That is,
\begin{equation}
\begin{aligned}
\frac{1}{2}J_{i0} (f,g) &=-\frac{1}{12d^2}[(-g_{i,1}-g_{i+1,1}+g_{i,0}+g_{i+1,0})(f_{i+1,0}+f_{i,0})\\
&-(-g_{i-1,1}-g_{i,1}+g_{i-1,0}+g_{i,0})(f_{i-1,0}+f_{i,0})\\
&+(g_{i+1,0}+g_{i+1,1}-g_{i-1,0}-g_{i-1,1})(f_{i,0}+f_{i,1})\\
&+(g_{i+1,0}-g_{i,1})(f_{i,0}+f_{i+1,1})\\
&+(g_{i,1}-g_{i-1,0})(f_{i-1,1}+f_{i,0})].
\end{aligned}
\end{equation}
Similarly, we obtain the scheme for boundary $j=J$:
\begin{equation}
\begin{aligned}
\frac{1}{2}J_{iJ} (f,g) &=-\frac{1}{12d^2}[(g_{i,J-1}+g_{i+1,J-1}-g_{i,J}-g_{i+1,J})(f_{i+1,J}+f_{i,J})\\
&-(g_{i-1,J-1}+g_{i,J-1}-g_{i-1,J}-g_{i,J})(f_{i-1,J}+f_{i,J})\\
&-(g_{i+1,J}+g_{i+1,J-1}-g_{i-1,J}-g_{i-1,J-1})(f_{i,J}+f_{i,J-1})\\
&-(g_{i+1,J}-g_{i,J-1})(f_{i,J}+f_{i+1,J-1})\\
&-(g_{i,J-1}-g_{i-1,J})(f_{i-1,J-1}+f_{i,J})].
\end{aligned}
\end{equation}
The discrete integral constraint $gf$ is $$\sum_{i}[ \frac{1}{2}g_{i,0}J_{i0}+\sum_{j=1}^{J-1} g_{i,j}J_{ij}+\frac{1}{2}g_{i,J}J_{iJ}].$$ %TODO: This is not even an equation.

\begin{remark}
For second order Arakawa scheme with g Dirichlet boundary condition, we can take $g_{i,0}=g_{i,-1}$ as a constant. However we can't directly modify fourth order scheme in this way by taking $g_{i,0}=g_{i,-1}=g_{i,-2}$ since the coefficients  $a_{i,0;i,-1}=\frac{1}{8d^2}(g_{i-1,1}-g_{i+1,0})$ and $a_{i,0;i+1,-1}=\frac{-1}{8d^2}(g_{i+1,1}-g_{i-1,0})$ (from $J^{\times \times}$) are not 0, so that we can't use half stencil. If we use half stencil, the coefficients can not be balanced, so we can't keep the conservation. 
\end{remark}

If function $f$ is constant at boundary: By $J_{ij}=-J_{ji}$, exchanging the position of f and g, we can get the scheme for the boundary f constant case:

for $j=0$:

\begin{equation}
\begin{aligned}
\frac{1}{2}J_{i0} (f,g) &=\frac{1}{12d^2}[(-f_{i,1}-f_{i+1,1}+f_{i,0}+f_{i+1,0})(g_{i+1,0}+g_{i,0})\\
&-(-f_{i-1,1}-f_{i,1}+f_{i-1,0}+f_{i,0})(g_{i-1,0}+g_{i,0})\\
&+(f_{i+1,0}+f_{i+1,1}-f_{i-1,0}-f_{i-1,1})(g_{i,0}+g_{i,1})\\
&+(f_{i+1,0}-f_{i,1})(g_{i,0}+g_{i+1,1})\\
&+(f_{i,1}-f_{i-1,0})(g_{i-1,1}+g_{i,0})].
\end{aligned}
\end{equation}

for $j=J$:

\begin{equation}
\begin{aligned}
\frac{1}{2}J_{iJ} (f,g) &=\frac{1}{12d^2}[(f_{i,J-1}+f_{i+1,J-1}-f_{i,J}-f_{i+1,J})(g_{i+1,J}+g_{i,J})\\
&-(f_{i-1,J-1}+f_{i,J-1}-f_{i-1,J}-f_{i,J})(g_{i-1,J}+g_{i,J})\\
&-(f_{i+1,J}+f_{i+1,J-1}-f_{i-1,J}-f_{i-1,J-1})(g_{i,J}+g_{i,J-1})\\
&-(f_{i+1,J}-f_{i,J-1})(g_{i,J}+g_{i+1,J-1})\\
&-(f_{i,J-1}-f_{i-1,J})(g_{i-1,J-1}+g_{i,J})].
\end{aligned}
\end{equation}




\subsubsection{Extrapolation at boundary for the 4th order Arakawa}

It is easy to get 4th order convergence, but how about conservation properties?

\textbf{Arakawa with homogeneous dirichlet boundary:}\\
In \cite{crouseilles2018exponential}, they show that the mass of Arakawa scheme for dirichlet boundary is not conserved up to machine precision, even for homogeneous Dirichlet boundary conditions. By their numerical experiments, the Arakawa scheme works better for homogeneous boundary condition. In their test, they assume that $f(r_{min}, \theta,z,v)=f_{eq}(r_{min},v)$ and $f(r_{max}, \theta,z,v)=f_{eq}(r_{max},v)$ but not homogeneous Dirichlet boundary. In addition to the direct formulation, we can also introduce a so-called perturbation formulation by dividing f into $f=f_{eq}+\delta f(t,r,\theta,v)$. With this formulation, our model problem can be written as 
\begin{equation}
 \partial_t \delta f + \{\phi, \delta f \} + v_\parallel \nabla_\parallel \delta f - \nabla_\parallel \phi\,\, \partial_{v_\parallel} \delta f + \{\phi, \delta f_{eq} \}= 0,
\end{equation} 
which is similar as our model problem. The Arakawa scheme that is used to discretize $\{\phi, \delta f \}$ now employs homogeneous Drichlet boundary conditions for $\delta f$ in r-direction.

% !TeX spellcheck = en_GB


\section{Numerical Experiments}
\label{sec:num_exp}


\subsection{Vortex Advection}

In order to verify the Arakawa scheme and it's properties, we look at a vortex advection example, where we perform the poloidal advection based on the potential $\phi$ and density $f$ given by 
\begin{align}
	\phi(r, \theta)  = 10 r, \quad f(r, \theta) = f_\text{eq}(r) + \frac14 e^{- ((\theta-\pi)^2 + (r -7)^2)},
\end{align}  
on a polar domain with $r \in [0.1, 14.5]$ and $\theta \in [0, 2\pi]$. This results in a banana-shaped Gaussian rotating around the centre of the disk. Since the value of the potential is directly proportional to the radius, the advection is stronger the closer the values are to the centre. Thus, resulting in a spiral as seen in \ref{fig:vortex}.

\begin{figure}[h]
	\centering
	\includegraphics[width=0.45\linewidth]{plots/vortex_init.png}
	\includegraphics[width=0.45\linewidth]{plots/vortex_final.png}
	\caption{The density $f$ at initial (left) and final (right) time.}
	\label{fig:vortex}
\end{figure}

In order to solve this test-problem, we use the Arakawa scheme of order $4$, with an extrapolation of the equilibrium function as a boundary condition, as a space discretization, for which we divide the angular domain in $N_\theta = 60$ and radial domain in $N_r = 40$ cells. The time integration is done by an explicit Runge-Kutta scheme of order $4$, where we perform $N = 200$ time-steps with a step-size of $\Delta t = 0.1$.
\begin{figure}[h]
	\centering
	\includegraphics[width=0.45\linewidth]{plots/vortex_cons.png}
	\includegraphics[width=0.45\linewidth]{plots/vortex_conv.png}
	\caption{Conservation of mass, $l^2$-norm and energy by the Arakawa scheme (left) and its order of convergence (right).}
	\label{fig:vortex_con}
\end{figure}
While solving, we keep track of the $l^1$-norm, or mass in other words, $l^2$-norm and the potential energy. As shown in \ref{fig:vortex_con}, these properties are preserved by the discrete solution as we have shown in section \ref{sec:consv-props}. Furthermore, we compare the relative error of our method on subsequently refined meshes in order to compute the order of convergence as shown in \ref{fig:vortex_con}.





\subsection{PyGyro}

Our main point of comparing the Arakawa method to the Semi-Lagrangian scheme are the conserved quantities: the mass
\begin{equation}
	\int f(r, \theta, z, v_\parallel) \d r \d \theta \d z \d v_\parallel
\end{equation}
and $l^2$-norm
\begin{equation}
	\left[\int f^2(r, \theta, z, v_\parallel) \d r \d \theta \d z \d v_\parallel\right]^\frac{1}{2}
\end{equation}
of the distribution function, and the potential energy
\begin{equation}
	\int \left(f(r, \theta, z, v_\parallel) - f_\text{eq}(r, v_\parallel)\right) \Phi(r, \theta, z) \d r \d \theta \d z \d v_\parallel
\end{equation}
, as well as the not necessarily conserved kinetic energy
\begin{equation}
	\int \left(f(r, \theta, z, v_\parallel) - f_\text{eq}(r, v_\parallel)\right) v^2_\parallel \d r \d \theta \d z \d v_\parallel
\end{equation}

From a simulation with grid size $[r:128, \theta:256, z:128, v_\parallel :72]$ and time step-size $\Delta t = 1$ we plot for the above mentioned 4 quantities the absolute error (figure \ref{fig:abserr}), the relative error (figure \ref{fig:relerr}), and the relative error on a logarithmic scale (\ref{fig:relerrlog}) before and after the poloidal advection step.

We can clearly see that the Arakawa scheme preserves the conserved quantities much better than semi-Lagrangian scheme. In the linear phase, the error is of order of machine precision. Only very late in the non-linear phase becomes the relative error bigger, but is still one or two orders of magnitude smaller than in the semi-Lagrangian scheme.

The kinetic energy is also much better preserved.

The same study with time-step size $\Delta t = 2$ was done (figures \ref{fig:abserr_dt2}, \ref{fig:relerr_dt2}, \ref{fig:relerrlog_dt2}) and yields the same results. It is therefore interesting to compare the results of the poloidal advection step for the Arakawa method for the two time step-sizes; these results are shown in figures \ref{fig:abserr_akw}, \ref{fig:relerr_akw}, \ref{fig:relerrlog_akw}. We see that the doubling of the time step-size results in double the absolute error for the conserved quantities, and about a factor of 5 for the kinetic energy.

%TODO: Only log scale probably
\begin{figure}
	\centering
	\includegraphics[width=0.9\linewidth]{plots/abs_err}
	\caption{The absolute error for different quantities before and after the poloidal advection step.}
	\label{fig:abserr}
\end{figure}


\begin{figure}
	\centering
	\includegraphics[width=0.9\linewidth]{plots/rel_err}
	\caption{The relative error for different quantities before and after the poloidal advection step.}
	\label{fig:relerr}
\end{figure}


\begin{figure}
	\centering
	\includegraphics[width=0.9\linewidth]{plots/rel_err_log}
	\caption{The relative error for different quantities before and after the poloidal advection step on a semi-logarithmic scale.}
	\label{fig:relerrlog}
\end{figure}


\begin{figure}
	\centering
	\includegraphics[width=0.9\linewidth]{plots/abs_err dt2}
	\caption{The absolute error for different quantities before and after the poloidal advection step with time step-size $\Delta t = 2$.}
	\label{fig:abserr_dt2}
\end{figure}


\begin{figure}
	\centering
	\includegraphics[width=0.9\linewidth]{plots/rel_err dt2}
	\caption{The relative error for different quantities before and after the poloidal advection step with time step-size $\Delta t = 2$.}
	\label{fig:relerr_dt2}
\end{figure}


\begin{figure}
	\centering
	\includegraphics[width=0.9\linewidth]{plots/rel_err_log dt2}
	\caption{The relative error for different quantities before and after the poloidal advection step on a semi-logarithmic scale with time step-size $\Delta t = 2$.}
	\label{fig:relerrlog_dt2}
\end{figure}


\begin{figure}
	\centering
	\includegraphics[width=0.9\linewidth]{plots/abs_err akw}
	\caption{The absolute error for different quantities before and after the poloidal advection step with the Arakawa method, comparing time step-size $\Delta t = 1$ and $\Delta t = 2$.}
	\label{fig:abserr_akw}
\end{figure}


\begin{figure}
	\centering
	\includegraphics[width=0.9\linewidth]{plots/rel_err akw}
	\caption{The relative error for different quantities before and after the poloidal advection step with the Arakawa method, comparing time step-size $\Delta t = 1$ and $\Delta t = 2$..}
	\label{fig:relerr_akw}
\end{figure}


\begin{figure}
	\centering
	\includegraphics[width=0.9\linewidth]{plots/rel_err_log akw}
	\caption{The relative error for different quantities before and after the poloidal advection step on a semi-logarithmic scale with the Arakawa method, comparing time step-size $\Delta t = 1$ and $\Delta t = 2$..}
	\label{fig:relerrlog_akw}
\end{figure}





\section{Conclusion}
\label{sec:conclusion}

Bonne travaille, je suis très satisfée.


\begin{acknowledgement}
	\textbf{Acknowledgements}\\
	The authors would like to thank Emmanuel Franck, Hélène Hivert, Guillaume Latu, Hélène Leman, Bertrand Maury, Michel Mehrenberger, and Laurent Navoret for organizing the CEMRACS conference 2022 and for the wonderful opportunity to come to Marseille and do research. We express special thanks to Michel Mehrenberger, Virginie Grandgirard, and Martin Campos Pinto for the daily supervision and general shaping of the project. Dominik Bell and Frederik Schnack thank Eric Sonnendrücker for the opportunity to participate in the CEMRACS and the fruitful discussions after the conference to give this project the final details. All authors are tremendously thankful for Emily Bourne and her help with the implementation and the work she did on the PyGyro code, which laid the ground-work for this project.
\end{acknowledgement}

\bibliographystyle{ieeetr}
\bibliography{literature}
\end{document}
